\documentclass{article}
\usepackage{graphicx} % Required for inserting images
\graphicspath{ {./images/} }

\setlength{\parindent}{0pt}

\title{Data Science Coursework 1}
\author{CID 01843091}
\date{February 2024}

\begin{document}

\maketitle

\section{Article}

Fracking is a technique that injects water, sand and other chemicals into a wellbore to create cracks in the deep-rock formations through which natural gas 
and petroleum can be extracted more efficiently. However, there have been health and environmental problems associated with fracking. Furthermore, Fracking 
activity in the USA has increased exponentially over the past decades, with a large proportion of it concentrated in a few states like New Mexico, Oklahoma
and Texas

\includegraphics[width=5cm, height=4cm]{./fracked_states_cropped.jpg}

It should be noted that FracFocus is funded partially by oil and gas trade groups, and has received skepticism on under-reporting. An earlier report by the 
federal Environmental Protection Agency found that between 2010 and 2012, 11 percent of the chemicals used in fracking were unreported. Konschnik and Dayalu’s 
study found that between 2012 and April 2015, that rose to 16.5 percent. By law, regulation regarding Fracking disclosure is managed by states and different 
states adopt different stances.

\includegraphics[width=5cm, height=4cm]{./disclosure_rules_cropped.jpg}

For many state with partial regulatory disclosure, disclosure is on a voluntary basis. They also have trade secret provisions that preclude companies from 
disclosing the exact chemical content of their fluids. This makes it difficult for states to assess the risk and appropriate measures to protect public health 
and safety. Navigating the advantages and drawbacks of Fracking poses a challenging dilemma with divergent perspectives. However, bolstering Fracking disclosure 
regulations enhances transparency and comprehension, making it a universally beneficial endeavor.


\end{document}